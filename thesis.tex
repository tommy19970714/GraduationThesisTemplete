%
%  論文
%
% (thesis.tex)
%%%%% for NTT TeX
\documentclass[a4j,12pt]{jreportTsukuba}
\usepackage{ascmac, color}%網掛け
\usepackage[dvipdfmx]{graphicx}
\usepackage{epsfig}
\usepackage{rotating}
\usepackage{lscape}
\usepackage{fancyheadings}
\usepackage[refpages]{gloss}
\usepackage{amsmath}
\usepackage{color}
\usepackage{enumerate}
\usepackage{slashbox}
\usepackage[T1]{fontenc}
\usepackage{textcomp}
\usepackage{amsmath,amssymb}
\usepackage{tabularx}
\usepackage{here}
\usepackage{comment}
\usepackage{subfigure}
\usepackage {threeparttable}

\makegloss % for glossary
\newcommand{\dg}{\gt}

\renewcommand{\chaptermark}[1]{\markboth{第 \thechapter 章\ \ #1}{}}
%\renewcommand{\chaptermark}[1]{\markboth{#1}{#1}} % \leftmark
\renewcommand{\sectionmark}[1]{\markright{\thesection\ #1}}
%\pagestyle{plain}
\makeatletter
\def\Hline{%
\noalign{\ifnum0=`}\fi\hrule \@height 2pt \futurelet
\reserved@a\@xhline}
\makeatother

\pagestyle{fancyplain}



%%%%% Option 1
\lhead[\fancyplain{}{\sf\thepage}]{\fancyplain{}{\rightmark}}
\rhead[\fancyplain{}{\bf\leftmark}]{\fancyplain{}{\sf\thepage}}
\cfoot[\fancyplain{\sf\thepage}{}]{\fancyplain{\sf\thepage}{}}

% for fancyheadings
\setlength{\footrulewidth}{0.4pt}

\setcounter{secnumdepth}{3}
\setcounter{tocdepth}{2}

%%% various length to be set
\headsep=0.5in
\parindent=1em
\topmargin = -0.54cm
\addtolength{\footskip}{0.2in}
\addtolength{\oddsidemargin}{-0.54cm}
\addtolength{\evensidemargin}{-0.54cm}
\addtolength{\textwidth}{1.08cm}
\addtolength{\headwidth}{1.08cm}
\topsep=0.1cm
\parsep=0.1cm
\itemsep=0.0cm
\renewcommand{\baselinestretch}{1.1}

%%% for Pen mark of Keio
%\newfont{\kten}{kiis10}
%\newcommand{\pen}{\kten p}
%%% for production rule
%\newcommand{\tate}{\,$\mid$\,} 
\newcommand{\ra}{\,$\rightarrow$\,}
\newcommand{\eps}{\,$\epsilon$\,}
\newcommand{\lkaku}{\,$\langle$\,}
\newcommand{\rkaku}{\,$\rangle$\,}
\newcommand{\gen}{\,$::=$\,}
\newcommand{\nonterm}[1]{\,$\langle$\,{\it #1}\,\,$\rangle$\,}
\newcommand{\terminal}[1]{\,$\langle$\,{#1}\,\,$\rangle$\,}
\newcommand{\bsl}{$\backslash$}
%%% for table with multicolumn
\newcommand{\lwr}[1]{\smash{\lower2.ex\hbox{#1}}}
%%% nqoute environment
\newenvironment{nquote}[1]%
{\list{}{\leftmargin=#1}\item[]}%
{\endlist}
\newcommand{\namelistlabel}[1]{\mbox{#1}\hfil}
\newenvironment{namelist}[1]{%
\begin{list}{}
  {\let\makelabel\namelistlabel
  \settowidth{\labelwidth}{#1}
  \setlength{\leftmargin}{1.1\labelwidth}}
}{%
\end{list}}
%%% for using musictex & \cup in math environment
% \cup is re-defined as crotchet in musictex (stupid!)
% rename \cup -> \mcup (defined in musictex.tex)
% redefine \cup as correct math symbol
\renewcommand{\cup}{\mcup}
\mathchardef\cup="225B

%%% here we go!
\begin{document}

\begin{titlepage}
\begin{huge}
\begin{center}
\vspace*{4cm}
{\huge {\bf 論文タイトル}}%bf:太字
\vspace{4cm}\\
2019年3月
\vspace{2cm}\\
\begin{tabular}{c}
学籍番号\\
山田 太郎
\end{tabular}
\vspace{2cm}\\
\begin{tabular}{c}
大学学群\\
学類
\end{tabular}
\end{center}
\end{huge}
\end{titlepage}

%\newtheorem{th}{{\dg 定義}}[section]
\renewcommand{\bibname}{参考文献}
%\renewcommand{\abstractname}{梗 概} 
%\renewcommand{\glossname}{用語集}

%% Title
%\maketitle
%% table of contents, figures, tables
\pagenumbering{roman}
%\listoffigures
%\listoftables
%% the body
\setlength{\baselineskip}{20pt}
%\renewcommand{\baselinestretch}{1.2}
%\pagenumbering{arabic}
%\cleardoublepage   %空白ページ
%%抄録
\documentclass[a4j,11pt]{jsarticle}
\thispagestyle{empty}

\usepackage{color}
\usepackage{ascmac}

%余白の設定************************
\setlength{\topmargin}{30mm}
\addtolength{\topmargin}{-1in}
\setlength{\oddsidemargin}{30mm}
\addtolength{\oddsidemargin}{-1in}
\setlength{\evensidemargin}{30mm}
\addtolength{\evensidemargin}{-1in}
\setlength{\headsep}{0mm}
\setlength{\headheight}{0mm}
\setlength{\topskip}{0mm}
\setlength{\textwidth}{138mm}
\setlength{\textheight}{215mm}
%**********************************

%背景    
%提案手法  
%実験    
%今後の課題 

\begin{document}
\begin{center}
\textgt{{\large 論文タイトル}}
\end{center}
\hspace{\fill}{\normalsize 山田太郎}\\[7mm]

アブストラクト内容

\vspace{\fill}\hspace{\fill}{\normalsize (指導教員 名前)}
\end{document}

\tableofcontents
\listoftables
\listoffigures

\cleardoublepage
\pagenumbering{arabic}
% Chapter 1
\chapter{はじめに}\label{chap1}
はじめにの内容

\section{背景と目的}\label{1-1}
背景と目的の内容

\section{論文の構成}\label{1-3}
論文の構成の内容
\chapter{関連研究}\label{chap2}

関連研究の内容
\chapter{提案手法}\label{chap3}

提案手法の内容
\chapter{実験}\label{chap4}
実験について

\section{実験の目的}\label{4-1}
実験の目的の内容

\section{実験データ}\label{4-2}
実験データの内容

\section{実験方法}\label{4-3}
実験方法の内容

\section{実験結果}\label{4-4}
実験結果の内容
\chapter{おわりに}\label{chap6}
\section{本研究のまとめ}
まとめの内容

\section{今後の課題}
今後の課題の内容
\chapter*{謝辞}
\addcontentsline{toc}{chapter}{謝辞}
\pagestyle{plain}
\pagestyle{fancyplain}

本研究を進めるにあたって,熱心かつ的確にご指導を頂いた,〇〇教授に深く感謝の意を表します.


加えて,本研究室の先輩である〇〇先輩や〇〇先輩には,ゼミや論文の添削の際に,様々なアドバイスを頂きました.
ここに感謝の意を表します. % 謝辞
\bibliographystyle{junsrt}
%\addcontentsline{toc}{chapter}{参考文献}
\bibliography{doc}
%\printgloss{myglos}

%% the appendix
\appendix % 付録
%\include{other/appndix1}
%\cleardoublepage
%\include{other/appndix2}
%\cleardoublepage
%\include{other/appndix3}
%\include{other/biblgrph} % 発表論文
\end{document}
